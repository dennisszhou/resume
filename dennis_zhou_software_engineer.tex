%%%%%%%%%%%%%%%%%%%%%%%%%%%%%%%%%%%%%%%%%%%%%%%%%%%%%%%%%%%%%%%%%%%%%%%%%%%%%%%%
% Medium Length Graduate Curriculum Vitae
% LaTeX Template
% Version 1.2 (3/28/15)
%
% This template has been downloaded from:
% http://www.LaTeXTemplates.com
%
% Original author:
% Rensselaer Polytechnic Institute 
% (http://www.rpi.edu/dept/arc/training/latex/resumes/)
%
% Modified by:
% Daniel L Marks <xleafr@gmail.com> 3/28/2015
%
% Important note:
% This template requires the res.cls file to be in the same directory as the
% .tex file. The res.cls file provides the resume style used for structuring the
% document.
%
%%%%%%%%%%%%%%%%%%%%%%%%%%%%%%%%%%%%%%%%%%%%%%%%%%%%%%%%%%%%%%%%%%%%%%%%%%%%%%%%

%-------------------------------------------------------------------------------
%   PACKAGES AND OTHER DOCUMENT CONFIGURATIONS
%-------------------------------------------------------------------------------

%%%%%%%%%%%%%%%%%%%%%%%%%%%%%%%%%%%%%%%%%%%%%%%%%%%%%%%%%%%%%%%%%%%%%%%%%%%%%%%%
% You can have multiple style options the legal options ones are:
%
%   centered:   the name and address are centered at the top of the page 
%               (default)
%
%   line:       the name is the left with a horizontal line then the address to
%               the right
%
%   overlapped: the section titles overlap the body text (default)
%
%   margin:     the section titles are to the left of the body text
%       
%   11pt:       use 11 point fonts instead of 10 point fonts
%
%   12pt:       use 12 point fonts instead of 10 point fonts
%
%%%%%%%%%%%%%%%%%%%%%%%%%%%%%%%%%%%%%%%%%%%%%%%%%%%%%%%%%%%%%%%%%%%%%%%%%%%%%%%%

\documentclass[margin,11pt]{res}  

% Default font is the helvetica postscript font
\usepackage{helvet}

% Increase text height
\textheight=700pt

\begin{document}

%-------------------------------------------------------------------------------
%   NAME AND ADDRESS SECTION
%-------------------------------------------------------------------------------
\name{Dennis Zhou}

% Note that addresses can be used for other contact information:
% -phone numbers
% -email addresses
% -linked-in profile

\address{US Citizen}
\address{dennis@kernel.org $\bullet$ 651-442-8757}

% Uncomment to add a third address
%\address{Address 3 line 1\\Address 3 line 2\\Address 3 line 3}
%-------------------------------------------------------------------------------

\begin{resume}

%-------------------------------------------------------------------------------
%   EXPERIENCE SECTION
%-------------------------------------------------------------------------------
\section{EXPERIENCE}
\employer{\textbf{Cruise Llc.}}
\location{San Francisco, CA}
\dates{01/2023 - Present}
\title{\sl{Staff Software Engineer}}
\begin{position}
Software/Hardware co-design on future programs\\
Supporting Linux development across Embedded Systems\\
Working on ML training cluster efficiency targeting ray job stragglers via cpu scheduling
\end{position}


\employer{\textbf{Snowflake Inc.}}
\location{San Mateo, CA}
\dates{10/2021 - 01/2023}
\title{\sl{Senior Software Engineer}}
\begin{position}
FoundationDB Performance/Storage Team\\
Prototyped an observability framework with prometheus and grafana on AWS
Contributed to blob storage management for unistore 
\end{position}

\employer{\textbf{Google Inc.}}
\location{New York, NY}
\dates{04/2020 - 09/2021}
\title{\sl{Software Engineer}}
\begin{position}
Remote Flash Team\\
Focused on I/O efficiency across the Google storage stack
\end{position}

\employer{\textbf{Facebook Inc.}}
\location{New York, NY}
\dates{07/2018 - 04/2020}
\title{\sl{Software Engineer}}
\begin{position}
Kernel Team\\
Linux kernel contributor - percpu memory allocator, cgroups, btrfs
\end{position}

\vspace{-24pt}
\employer{}
\location{}
\dates{05/2017 - 08/2017}
\title{\sl{Software Engineer Intern}}
\begin{position}
Kernel Team\\
Linux kernel contributor - percpu memory allocator
\end{position}

\employer{\textbf{University of Wisconsin}}
\location{Madison, WI}
\dates{01/2017 - 05/2018}
\title{\sl{Research Assistant}}
\begin{position}
Worked with the Microsoft Gray Systems Lab on Sqlserver telemetry on Azure
\end{position}

\vspace{-24pt}
\employer{}
\location{}
\dates{09/2016 - 12/2016}
\title{\sl{Teaching Assistant}}
\begin{position}
CS 537 – Introduction to Operating Systems
\end{position}

%-------------------------------------------------------------------------------

%-------------------------------------------------------------------------------
%   PROGRAMMING SECTION
%-------------------------------------------------------------------------------
\section{PROGRAMMING}

\textbf{Proficient}: C, Java, Python\\
\textbf{Familiar}: C++, Perl\\
\textbf{Linux}: Tree-maintainer of Linux's percpu memory allocator since v4.19\\
https://git.kernel.org/pub/scm/linux/kernel/git/dennis/percpu.git
%-------------------------------------------------------------------------------

%-------------------------------------------------------------------------------
%   EDUCATION SECTION
%-------------------------------------------------------------------------------
\section{EDUCATION}
\textbf{University of Wisconsin}, Madison, WI\hfill 09/2016 - 05/2018\\
{\sl Master of Science}, Computer Sciences, Overall GPA: 4.0\\
Advised by Professors Remzi and Andrea Arpaci-Dusseau

\vspace{-10pt}
\textbf{Georgia Institute of Technology}, Atlanta, GA\hfill 08/2011 - 05/2014\\
{\sl Bachelor of Science}, Computer Science, Major GPA: 4.0\\
Threads: Information Internetworking and Systems \& Architecture
%-------------------------------------------------------------------------------

%-------------------------------------------------------------------------------
%   ACTIVITIES SECTION
%-------------------------------------------------------------------------------
\section{ACTIVITIES}
Enjoys outdoor activities including: snowboarding, golf, climbing, and tennis.\\
RPM Raceway Jersey City, NJ - \textbf{Fastest Lap Nov 2021}
%-------------------------------------------------------------------------------

\end{resume}
\end{document}
